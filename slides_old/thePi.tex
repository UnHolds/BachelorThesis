\begin{frame}[allowframebreaks]
	\frametitle{Computer?}
	\framesubtitle{Das passiert in der Hardware, wenn ich einen Befehl
	eingebe.}

	\begin{block}{Die Frage ist doch:}
		Muss man wissen was in der Hardware passiert?
	\end{block}
	\note{
		Wissen muss man es nicht immer.
		Genau dafür gibt es die Abstraktion:
		Man muss nicht alles wissen.
		Aber: öfters ist es gut um einen Fehler zu finden, etwas selber zu
		entwerfen oder die Leistung zu steigern
	}

	\framebreak

	\begin{block}{Machen wir ein kleines Hardwareprojekt und schauen nach!}
		Dabei werden wir
		\begin{itemize}
			\item sehen dass doch viel mehr hinter einem kleinen Gadget
				steckt als man denken mag.
			\item einen ersten Eindruck gewinnen, wie ein Computer aufgebaut ist und was denn so alles
				passiert wenn man damit interagiert.
			\item uns punktuell anschauen welche Kompetenzen man im Informatikstudium, am Beispiel
				der Technischen Informatik, erwirbt.
		\end{itemize}
	\end{block}
\end{frame}

\section{Das Projekt}
\begin{frame}
	\frametitle{Projekt Handy}

	\note{
				% Die Notes sind mehr als Aussagen/Gedanken des Vortragenden zu
				% sehen.
		Es wird zwar kein Smartphone, ist aber noch immer ein Computer im
		eigentlichen Sinne.
	}


	\pause

	\begin{center}
		Was brauchen wir dafür?
	\end{center}
\end{frame}

\begin{frame}
	\frametitle{Projekt Handy --- Display}

	\vspace{-0.5cm}

	\begin{itemize}
		\item \SI{2.8}{\inch} TFT + resistiver Touchscreen.
		\item 320$\times$240 Pixel mit je 16 Bit Farbinformation.
		\item ansprechbar über SPI Interface.
	\end{itemize}
\end{frame}

\begin{frame}
	\frametitle{Projekt Handy --- Display}

	\begin{itemize}
		\item Jedes Computersystem hat in irgendeiner Form eine Interaktion mit
			dem Benutzer.
		\item Das kann im einfachsten Fall  Taster + LED sein.
		\item In unserem Fall ist ein Touchscreen im Einsatz.
	\end{itemize}
\end{frame}

\begin{frame}
	\frametitle{Projekt Handy --- Mobilfunk}
			% Wir brauchen eine Anbindung an das Mobilnetz

%\end{frame}

	\vspace{-0.5cm}

%\begin{frame}
%	\frametitle{Projekt Handy -- Mobilfunk}
	\begin{itemize}
		\item SIM900 Quad-Band GSM/GPRS Module.
		\item ansprechbar über serielle Schnittstelle und AT-Commands.
	\end{itemize}
\end{frame}

\begin{frame}
	\frametitle{Projekt Handy --- Mobilfunk}

	\begin{block}{Computer kommunizieren und interagieren typischerweise}
		\begin{itemize}
			\item mit dem Benutzer\\
			per ``User Interface''
			\item \emph{mit anderen Computern} \\
			``Vernetzung'', ``verteilte Systeme''
			\item mit der Umwelt\\
			 per Sensor/Aktor, ``Embedded Systems''
		\end{itemize}
	\end{block}
\end{frame}

\begin{frame}
	\frametitle{Projekt Handy --- Steuerung}
%\end{frame}
\vspace{-0,5cm}
%\begin{frame}
%	\frametitle{Projekt Handy -- Steuerung}
	\begin{itemize}
		\item Broadcom BCM2835 @ \SI{700}{\mega\hertz} ARM11 core,
			\SI{256}{\mega\byte}  RAM.
		\item 26 Pin Header für Erweiterungen \ldots
		\item bietet SPI und serielle Schnittstelle.
	\end{itemize}
\end{frame}

\begin{frame}
	\frametitle{Projekt Handy --- Steuerung}

	\begin{itemize}
		\item Macht die ``Intelligenz'' im System aus.
		\item Ermöglicht die Programmierung entsprechend der Anwendung (Handy, Alarmmelder, Anrufbeantworter, \ldots).
		\item Führt passende Ansteuerung von Touchscreen und Modem durch.
		\item Macht aus der Summe der Teile ein Handy.
	\end{itemize}

\end{frame}

\begin{frame}
	\frametitle{Projekt Handy --- Energie}
	Jeder Computer braucht eine Energiequelle; \\
	in unserem Fall soll sie portabel sein.
	\vspace{-0.5cm}
	\begin{itemize}
		\item Akkupack mit integrierter Ladeelektronik.
		\item \SI{2200}{mAh}
	\end{itemize}
\end{frame}

\begin{frame}
	\frametitle{Projekt Handy --- Kostenaufstellung}
	\note{Kicken?}
	\begin{center}
		\begin{tabular}{rl} \toprule
			Preis (\euro) & Komponente \\ \midrule
			35 & Raspberry Pi\\
			35 & Adafruit PiTFT\\
			40 & EFCom GPRS/GSM Module\\
			15 & Akku\\
			5 & Kleinteile\\\cmidrule(lr){1-2}
			130 & Gesamt\\
			\bottomrule
		\end{tabular}
	\end{center}
\end{frame}

\liveDemo{Projekt Handy --- Putting it all together}

\begin{frame}
    \centering
	\begin{block}{Quiz:}
		Wie nennt man die kleinste Informationseinheit, die ein Computer
		verarbeitet?
		\begin{enumerate}
			\item PIN
			\item BIT
			\item LED
			\item PUK
		\end{enumerate}
	\end{block}

	\medskip

	\uncover<2->{
		\huge
		\thetelefonnumber
	}
\end{frame}

\begin{frame}
	\frametitle{Was wissen wir eigentlich über \ldots}
			% Prüfungsfach Programmierung

			% Ansteuerung des Displays und Interaktion mit dem Modem

	\begin{block}{\ldots den Rechner?}
		\begin{itemize}
			\item Ein Rechner umfasst Prozessor, Speicher und Peripherie.
			\item Der Prozessor versteht Befehle und kann daher programmiert werden.
			\item Nicht jeder Prozessor ist gleich, je nach Hersteller versteht er
				andere Befehle.
			\item Wie der Prozessor auf Hardware Ebene genau funktioniert, werden
				wir uns morgen genauer ansehen.
			\item Für unser Projekt ist uns das egal, durch die Verwendung
				einer \alert<2->{Programmiersprache} abstrahieren wir das
				alles weg und \alert<2->{arbeiten prozessorunabhängig} (soweit
				es geht).

		\end{itemize}
	\end{block}
\end{frame}

\begin{frame}
			% Abstraktion: Wie der Touch funktioniert ist egal, wir bekommen von
			% einem Treiber nur gesagt wo gedrückt wurde und reagieren darauf

	\begin{block}{\ldots den Touchscreen?}
		\begin{itemize}
			\item Wie wird der Druck des Fingers erkannt?
			\item Durch Änderung der Kapazität an einer bestimmten Koordinate
			\item oder durch einen Spannungsabfall.

			\item Aber: Wie genau, ist uns eigentlich egal (Abstraktion!).
			\item \alert<2->{Wir bekommen nur die Information wo gedrückt wurde.}
		\end{itemize}
	\end{block}
\end{frame}

\begin{frame}
	\begin{block}{\ldots das Display?}
		\begin{itemize}
			\item Wie kommt ein Zeichen auf das Display?
			\item Spannung steuert, wie ein Flüssigkristall die Polarisationsrichtung von Licht beeinflusst.
			\item \alert<2->{Wir sagen nur, was wo am Display sein soll.}
		\end{itemize}
	\end{block}
\end{frame}


\begin{frame}

	\begin{block}{\ldots das Modem?}
		\begin{itemize}
			\item Wie funktioniert eigentlich die Funkübertragung?
			\item Über kontrollierte (modulierte) elektromagnetische
				Wellenausbreitung.
			\item Wie wandelt das Modem digitale Daten in elektromagnetische
				Wellen um?
			\item Wieder abstrahieren wir diese ``Details'' weg: \alert<2->{wir
					übergeben dem Modem Daten zum Senden.}
		\end{itemize}
	\end{block}
%	Stellt eine Schnittstelle zwischen dem analogen Mobilfunk und der
%	digitalen Signalverarbeitung dar.
%	Genaue funktionsweise wird im ET-Studium vermittelt, im BTI werden nur
%	Grundlagen der Elektrotechnik gebracht.
%	Jedoch ist das Modem ein ASIC (Application Specific Integrated
%	Ciruit) die sehr wohl behandelt werden.

			% Abstraktion: zwei Kabel reichen aus um viel mehr zu steuern
			% Experimentreihe: Das Kabel
\end{frame}

\begin{frame}
			% Prüfungsfach Signale und Systeme

	\begin{block}{\ldots das Audiointerface?}
		\begin{itemize}
	%		\item Neben Datendiensten ist die Sprachtelefonie noch immer eine wichtige Funktion eines Mobiltelefons
			\item Wie kommt die Sprache eigentlich von A nach B?
			\item Ein Mikrofon wandelt die Schallwellen in ein elektrisches
				Signal um.
			\item Und wie erhalten wir aus diesem elektrischen Signal einen Datenwert, der
				weiter verarbeitet werden kann?
			\item Dazu gibt es eigene Analog/Digital Konverter, die nach
				verschiedenen Prinzipien arbeiten.
			\item Doch auch diese Details nehmen uns darunter liegende Ebenen
				(HW Module, Treiber) ab, \alert<2->{wir betrachten die Signale
				einfach als digitale Daten, die wir verarbeiten.}
		\end{itemize}
	\end{block}
	\note{
		Behandlung der Daten wird von Chip gemacht; erleichtert die Arbeit
		Aliasing Demo?
	}
\end{frame}
