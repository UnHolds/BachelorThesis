\begin{frame}
	\frametitle{Fahrplan}

	\begin{block}{Was werden wir heute machen?}
		\begin{itemize}
			\item Einen Einblick in eines der wichtigsten Werkzeuge eines
				Technikers geben: die Abstraktion.
			\item Dabei anhand von Beispielen aufzeigen, dass Abstraktionen
				Grenzen haben und spannendes Verhalten verstecken kann.
			\item Einen ersten Eindruck in das Entstehen eines Computerchips
				und die dabei auftretenden Probleme geben.
		\end{itemize}
	\end{block}
\end{frame}

\begin{frame}
	\frametitle{Abstraktion --- Was ist das?}

	\begin{block}{Abstraktionen sind überall!}
		\begin{itemize}
			\item Eine Abstraktion ist ein Modell.
			\item Dieses Modell versteckt komplexes Verhalten durch Weglassen
				von Informationen/Details.
			\item Dadurch ist ein Sachverhalt leichter zu verstehen, als durch die
				Summe der zugrundeliegenden Modelle.
				% In Software: Module
				% In Hardware: Komponenten
				% Erdbeschleuigung ~9.81; jedoch 0.5% zwischen Äquator und den Polen!
			\item Abstraktionen haben aber auch ihre Grenzen!
		\end{itemize}
	\end{block}

\end{frame}

\begin{frame}
	\frametitle{Abstraktion --- Beispiel: analog vs.\ digital}

	\begin{exampleblock}{Was bedeutet logisch 1 bzw.\ logisch 0?}
		Abstraktion des gemessenen Signals

		\begin{columns}[c]
			\begin{column}{.45\textwidth}
				\begin{figure}[h]
					\centering
					\begin{tikzpicture}
						\pgfplotsset{set layers=standard}
						\pgfplotsset{width=7cm}
						\pgfplotsset{colormap={hot}{color(0cm)=(blue); color(1cm)=(red)}}
						\begin{axis}[xlabel={t},
								ylabel={Spannung},
								xminorticks=false,
								xmajorticks=false,
								yminorticks=false,
								ymajorticks=false,
								axis x line=center,
								axis y line=left,
								ymin=-.1,
								ymax=1.1,
							]
							\pgfonlayer{axis background}
								\fill[color=blue!20!white] (axis cs:0,-0.1) -| (axis cs:1,0.4) -| cycle;
								\fill[color=red!20!white] (axis cs:0,1.1) -| (axis cs:1,0.6) -| cycle;
								\node at (axis cs: .1, .9) {log.\ 1};
								\node at (axis cs: .1, .1) {log.\ 0};
							\endpgfonlayer
							\addplot[mesh, point meta=y] table {plotdata/real_step.dat};
						\end{axis}
					\end{tikzpicture}
				\end{figure}
			\end{column}
			\begin{column}{.45\textwidth}
				\begin{figure}[h]
					\centering
					\begin{tikzpicture}
						\pgfplotsset{set layers=standard}
						\pgfplotsset{width=7cm}
						\pgfplotsset{colormap={hot}{color(0cm)=(blue); color(5mm)=(black); color(1cm)=(red)}}
						\begin{axis}[xlabel={t},
								ylabel={Spannung},
								xminorticks=false,
								xmajorticks=false,
								yminorticks=false,
								ymajorticks=false,
								axis x line=center,
								axis y line=left,
								ymin=-.1,
								ymax=1.1,
							]
							\pgfonlayer{axis background}
								\fill[color=blue!20!white] (axis cs:0,-0.1) -| (axis cs:1,0.4) -| cycle;
								\fill[color=red!20!white] (axis cs:0,1.1) -| (axis cs:1,0.6) -| cycle;
								\node at (axis cs: .1, .9) {log.\ 1};
								\node at (axis cs: .1, .1) {log.\ 0};
							\endpgfonlayer
							\addplot[mesh, point meta=y, thick, line cap=round] table {plotdata/ideal_step.dat};
						\end{axis}
					\end{tikzpicture}
				\end{figure}
			\end{column}
		\end{columns}

	\end{exampleblock}
\end{frame}

\begin{frame}
	\frametitle{Abstraktion --- Beispiel: Handy}

	\begin{exampleblock}{Handy}
		Wir können damit Anrufe tätigen und vieles mehr.
		\begin{itemize}
			\item Von der Übertragung der Daten zwischen Handy und Mast
				bemerken wir nichts. Allerdings ist hier komplizierte
                                Hochfrequenztechnik im Einsatz.
			\item Die Übertragung der Sprache passiert einfach. Warum und wie
				ist beim Telefonieren eigentlich egal, oder?
			\item Wie das Handy eine Verbindung zu einem
				Gesprächsteilnehmer aufbaut ist uns als Anwender auch egal:
                                Wir wählen nur eine Nummer.
			\item Wie reagiert das Handy auf unsere Eingabe? Was
				passiert bei einem Tastendruck?
		\end{itemize}
		\uncover<2->{
			\medskip
			Wir sehen schon mehrere Abstraktionen:
			\begin{itemize}
				\item Was ist die ``übermittelte Sprache'' eigentlich genau?
				\item Wie/wovon wird das Handy gesteuert?
			\end{itemize}
		}
	\end{exampleblock}

	%\begin{exampleblock}{Ladevorgang vom Webserver}
	% jetzt könnten wir runter gehen bis auf Elektronen die sich bewegen. Doch
	% wenn wir uns beim Laden einer Website über Elektronen gedanken machen
	% müssten, würden nicht viele Leute überhaupt einen Browser bedienen
	% können.
\end{frame}

